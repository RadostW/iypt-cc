(% block header %)
\pagestyle{empty}
\usepackage{multicol}
\parindent0pt
\parsep2ex
(% endblock %)

(% block geometry %)right=2cm, left=2cm, top=3.5cm, bottom=1cm(% endblock %)

(% block content %)

	(% for team in teams -%)
		\begin{center}
    {\boldfont 30\textsuperscript{th} International Young Physicists’ Tournament\medskip

    2017}\medskip

    Please state which problem you select for the (( round ))\textsuperscript{th} PF!\bigskip

\end{center}

\setlength\extrarowheight{10pt}

\begin{tabular}{llL{8cm}}
    \textbf{Team} & \textbf{(( team ))} & \\
    \hline
    1\textsuperscript{st} choice: & No.\ \ \ \ \ \ \ \ , &\\\hline
    2\textsuperscript{nd} choice: & No.\ \ \ \ \ \ \ \ ,&\\\hline
    3\textsuperscript{rd} choice: & No.\ \ \ \ \ \ \ \ ,&\\\hline
\end{tabular}\bigskip

{\footnotesize
{\blackfont IYPT-Regulations section IX/2 (Version 2015):}
 The following special rules apply to the last Selective PF:

The  procedure  of  challenge  is  omitted.  All  teams  may  choose  the  problem  to
present. The only exception is that a team may not present
a problem, which they
presented earlier in the Selective Fights, and all problems presented in one group
must be different. In case teams of one group choose the same problem, priority
is given to the team with the higher
TSP (see section XI).

A
team has to select its problems and submit this selection before it leaves the
competition  room  after  the  previous  PF.
The  choice  must  be  made  public
immediately.


The problem which a team presents in this PF may not be presented again in the
Final PF by the
same team.}

\begin{center}
 Please give this form
 to one of the organizers as soon as possible
 !
 \end{center}\bigskip

 \noindent\begin{tabular}{l}
 \makebox[3in]{\hrulefill}\\
 Signature Team Captain \\
 \end{tabular}\bigskip


Problems of the 30\textsuperscript{th} IYPT:
 \begin{multicols}{3}
	(% for p in problems %)
    	(( p[0] )). (( p[1] ))
    (% endfor %)
\end{multicols}
        \newpage
	(% endfor %)

(% endblock %)